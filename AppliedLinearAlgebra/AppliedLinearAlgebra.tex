\documentclass{article}
\usepackage{graphicx}
\usepackage{amsfonts}
\usepackage{amsmath}
\usepackage{mathtools}

\newcommand\setItemnumber[1]{\setcounter{enumi}{\numexpr#1-1\relax}}

\title{Applied Linear Algebra Notes}
\author{Joe Hollander}
\date{Summer 2025}  

\begin{document}
\maketitle 

\section{}
    \subsection*{1.1}
    Elementary Row Operations
    \begin{enumerate}
        \item Interchange two rows.
        \item Multiply the elements of a row by a nonzero constant.
        \item Add a multiple of the elements of one row to the corresponding elements of another.
    \end{enumerate}
    Gauss-Jordan elimination is a systematic way to eliminate variables to arrive at a solution matrix. 
    The equation is solved when the matrix is diagonalized. Create zeros in each column until finished. 
    Finally, normalize the final element ((3,3) for a 3x3 matrix, etc.). 
    Interchange rows if a diagonal element is zero. Final matrix is called the reduced echelon form.
    The formula can be used for multiple constants with the same coefficient matrix. 
    To do this, just stack the constants in each column: 
    \begin{gather}
    x - y + 3z = b_1 \\
    2x - y + 4z = b_2 \\
    -x + 2y -4z = b_3
    \end{gather}
    
    for 
    \[
    \begin{bmatrix}
        b_1 \\
        b_2 \\
        b_3 \\
    \end{bmatrix}
    =
    \begin{bmatrix}
        8 \\
        11 \\
        -11 \\
    \end{bmatrix},
    \begin{bmatrix}
        0 \\
        1 \\
        2 \\
    \end{bmatrix},
    \begin{bmatrix}
        3 \\
        3 \\
        4 \\
    \end{bmatrix}
    \]
    would produce the augmented matrix
    \[
    \begin{bmatrix}
        1 & -1 & 3 & 8 & 0 & 3 \\
        2 & -1 & 4 & 11 & 1 & 3 \\
        -1 & 2 & -4 & -11 & 2 & 4 \\
    \end{bmatrix}
    \]
    which simplifies to 
    \[
    \begin{bmatrix}
        1 & 0 & 0 & 1 & 0 & -2 \\
        0 & 1 & 0 & -1 & 3 & 1 \\
        0 & 0 & 1 & 2 & 1 & 2 \\
    \end{bmatrix}.
    \]

    \subsection*{1.2}
    A matrix is in reduced echelon form if:
    \begin{enumerate}
        \item Any rows consisting entirely of zeros are grouped at the bottom of the matrix.
        \item The first nonzero element of each other row is 1. This element is called a leading 1. 
        \item The leading 1 of each row after the first is positioned to the right of the leading 1 of the previous row.
        \item All other elements in a column that contain a leading 1 are zero. 
    \end{enumerate}
    The reduced echelon of a matrix is unique. 
    To express multiple solutions, write the leading varaibles in each equation 
    (variables with the lowest subscript: $x_1$, etc.)  in terms of the remaining free variables.
    If last row has all zeros apart from constant, there are no solutions for the system.
    A system of linear equations is said to be homogeneous if all the constants are zero.  
    A homogeneous system of linear equations always has a solution of all zeros. 
    A homogenous system of linear equations that has more variables than equations (more rows than columns) 
    has many solutions. 

    \subsection*{1.7}
    Curve fitting a polynomial involves estimating variables, which can be done by solving a system of linear equations.
    Kirchhoff's Laws state:
    \begin{enumerate}
        \item All the current flowing into a junction must flow out of it
        \item The sum of the IR terms (I denotes current, R resistance) in any direction around a closed path
        is equal to the total voltage in the path in that direction. 
    \end{enumerate}
    Using these laws a system of linear equations can be formed to determine the current flowing through certain points.
    The same procedure can be used with traffic. 

    \subsection*{1.3}
\end{document}