\documentclass{article}
\usepackage{graphicx}
\usepackage{physics}
\usepackage{amsfonts}
\usepackage{amsmath}
\usepackage{mathtools}
\usepackage{array, esint}

\newcommand\setItemnumber[1]{\setcounter{enumi}{\numexpr#1-1\relax}}

\title{Applied Linear Algebra Notes}
\author{Joe Hollander}
\date{Summer 2025}  

\begin{document}
\maketitle 

\begin{enumerate}
    \item 1.1
    Elementary Row Operations
    \begin{enumerate}
        \item Interchange two rows
        \item Multiply the elements of a row by a nonzero constant
        \item Add a multiple of the elements of one row to the corresponding elements of another
    \end{enumerate}
    Gauss-Jordan elimination is a systematic way to eliminate variables to arrive at a solution matrix. 
    The equation is solved when the matrix is diagonalized. Create zeros in each column until finished. 
    Finally, normalize the final element ((3,3) for a 3x3 matrix, etc.). 
    Interchange rows if a diagonal element is zero. Final matrix is called the reduced echelon form.
\end{enumerate}

\end{document}