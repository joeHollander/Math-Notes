\documentclass{article}
\usepackage{graphicx}
\usepackage{physics}
\usepackage{amsfonts}
\usepackage{amsmath}
\usepackage{mathtools}
\usepackage{array, esint}

\title{Linear Algebra Notes}
\author{Joe Hollander}
\date{Winter 2024}  

\begin{document}
\maketitle 

\newcommand{\tdv}[2]{
    \begin{bmatrix}
        #1 \\
        #2
    \end{bmatrix}}

\newcommand{\iltdv}[2]{
    \begin{bsmallmatrix}
        #1 \\
        #2
    \end{bsmallmatrix}}

In order to change basis into terms of two orthogonal vectors $b_1 = \iltdv{a}{b}$ and $b_2 = \iltdv{c}{d}$, 
solve this equation:
\[
\mathbf{r} = x\tdv{a}{b} + y\tdv{c}{d},
\]
where the new vector will be $\iltdv{x}{y}$. 
First, find  the dot product of $\mathbf{r}$ with both $b_1$ and $b_2$,
and then, divide by the square of the magnitude of $b_1$ and $b_2$ respectively to find $x$ and $y$.

Einstein notation simplifies
\[
\sum_{n}^{} A_in B_nj 
\]
to \[A_i B_j.\]

Applying a transformation $R$ for a different basis set $B$:
\[
B^-1 RB.
\]
A basis set is considered orthonormal if all bases are orthogonal and have a magnitude of 1.
A matrix is orthogonal if all of its columns are orthonormal bases.
Additionally, an orthogonal matrix has the property that $A^{-1} = A^T$ since $A^TA = I$.
The Gram-Schmidt process turns linearly independent vectors that span the plan
into orthonormal bases by substracting each vector by its projection onto the previous vectors
and then normalizing. 

Eigenvectors are vectors that don't change direction after a transformation,
and the eigenvector of a rotation is also its axis of rotation.
Eigenvalues are the values that the eigenvectors are scaled by.
This can be represented algebraically:
\[
Ax = \lambda x \implies (A - \lambda I)x = 0 \implies \det(A - \lambda I) = 0.
\]   
The characteristic polynomial is the determinant of the matrix $A - \lambda I$.

A diagonalized matrix is much easier for repeated matrix multiplication.
For example,
\[
T^n = 
\begin{bmatrix} 
    a^n & 0 & 0 \\
    0 & b^n & 0 \\
    0 & 0 & c^n
\end{bmatrix}.
\]
A matrix made of eigenvectors is called a modal matrix,
and a diagonalized matrix can be formed using the eigenvalues. 
Applying the eigenvector transformation and then the eigenvalues matrix
allows us to compute easily. For example, 
\begin{gather}
    T = CDC^{-1}, \nonumber \\ 
    T^2 = CDC^{-1}CDC^-1 = CD^2C^{-1}, \nonumber \\
    T^n  = CD^nC^{-1}. \nonumber
\end{gather}

\end{document}