\documentclass{article}
\usepackage{graphicx}
\usepackage{physics}
\usepackage{amsfonts}
\usepackage{amsmath}

\title{Investment Management Course Notes}
\author{Joe Hollander}
\date{Summer 2024} 

\begin{document}
\maketitle

\section[1.1]{Fundamentals of risk and returns}

Compounding returns with different return rates:
\[
(1 + r_1)(1 + r_2) - 1.
\]
Compounding returns with the same return rate:
\[
((1 + r)^t - 1). 
\]
To compare different time period standard deviations multiply (or divide)
by the square root of the number of time periods.
The sharpe ratio:
\[
\frac{R_p-R_f}{\sigma_p}.
\]
Pandas standard deviation method uses the sample
standard deviation and not the population standard
deviation. Similar to the sharpe ratio, the calmar
ratior is a risk adjusted return where risk is measured
by drawdown. Use index method to\_period to convert from
datetime to period. Use series method cummax to find the
highest value for each timestep. Drawdown is then: 
\[
\frac{Value - Previous Peak}{Previous Peak}.
\]


 


\end{document}