\documentclass{article}
\usepackage{amsfonts}
\usepackage{amsmath}
\usepackage{siunitx}
\usepackage{tabularx}

\newcommand{\mps}{$\si{m\per\second^2}$}

\title{Force of the Fan Cart Summative}
\author{Joe Hollander}
\date{Nov 2024}

\begin{document}
\maketitle 


Two settings, the second and third level, were used for all masses. Four masses were used:
the fan cart by itself, with a 0.25 kg bar, with two 0.25 kg bars,
and with the unknown mass. 
Position data was measured with a Vernier motion detector laying down on the track. 
The acceleration was measured for each mass and setting by fitting a quadratic equation
(because the fan cart has a constant acceleration)
to the position data and doubling the $x^2$ coefficient. 
Newton's second law states that $\mathbf{F} = m \mathbf{a}$, so each force was found by multiplying
the calculated acceleration by the mass in kg (newtons are in kgs). 
The force for each setting was calculated for each weight, and the forces were averaged to 
reduce the effect of outliers. 
The final averaged force for each setting was then used to calculate the unknown given its acceleration.
Rearranging Newton's second law to solve for mass: $m = \mathbf{\frac{F}{a}}$. 
Since, the high setting's forces varied greatly, the average medium setting force was used to estimate the mass by
dividing the average medium setting force by the acceleration. 
This mass was then subtracted from the original fan cart mass to find the calculated mass of the unknown object.

Fan cart number seven was used, and its mass was 531.86 $\si{g}$. 
The following values are all accelerations in \mps, formatted by mass, setting: trials.

\medskip

\noindent
531.86 g, Medium trials: 0.2452, 0.2404, 0.2396 \\
531.86 g, High trials: 0.4042, 0.4212, 0.3988, 0.3766 \\
781.86 g, Medium trials: 0.1192, 0.1326, 0.1436, 0.1363, 0.1460 \\
781.86 g, High trials: 0.2580, 0.2086, 0.1804, 0.2076, 0.1881 \\
1031.86 g, Medium trials: 0.0981, 0.1079, 0.1054, 0.1149, 0.1138 \\
1031.86 g, High trials: 0.1642, 0.1375, 0.1352, 0.1310, 0.1390 \\
Unknown mass, Medium trials: 0.1859, 0.1786, 0.1824, 0.1816, 0.1851 \\
Unknown mass, High trials: 0.2110, 0.2166, 0.2086, 0.2050, 0.2008
\medskip

The following table's entries are the averaged accelerations for each mass and setting. 

\begin{center}
    \begin{tabularx}{0.8\textwidth}{
        | >{\centering\arraybackslash}X 
        | >{\centering\arraybackslash}X 
        | >{\centering\arraybackslash}X 
        | >{\centering\arraybackslash}X 
        | >{\centering\arraybackslash}X | }
        \hline
        & 531.86 g & 781.86 g & 1031.86 g & Unknown Mass \\
        \hline
        Medium & 0.2417 \mps & 0.1355 \mps & 0.1080 \mps & 0.1827 \mps \\
        \hline
        High & 0.4002 \mps & 0.2085 \mps & 0.1414 \mps & 0.2084 \mps \\
        \hline
    
    \end{tabularx}
\end{center}

As an example, the force of the medium setting, using the 531.86 g acceleration, would be found 
by multiplying 0.2417 \mps \ by 0.53186 kg (531.86 g divided by 1000), which is 0.1286 N. 
The force values found for the medium setting were 0.1286 N, 0.1059 N, and 0.1114 N, 
while the force values found for the high setting were 0.2129 N, 0.1630 N, and 0.1459 N. 
The average of the medium setting values is 
\[
\frac{0.1286 + 0.1059 + 0.1114}{3} = \mathrm{0.1153 N}.
\]
Because the high setting forces varied a lot more, the data was omitted, 
and the average medium setting force was used to predict the unknown mass. 
Therefore, the fan cart with the unknown mass was calculated to have a mass of
\[
\frac{\mathrm{0.1153 \ N}}{0.1827 \ \si{m\per\second^2}} = \mathrm{0.6311 \ kg}.
\]
Subtracting this mass from the original fan cart mass of 0.53186 kg,
the unknown mass was calculated to be 0.09924 kg, or 99.24 g.

The actual mass of the unknown object was 148.53 g, so the percent error is
\[
\frac{148.53 - 99.24}{148.53} = 0.331 = 33.1\%. 
\]
The percent error is quite high. 
The error could be explained by not accounting for friction, outliers in the data, or
damage to the fan cart after getting dropped. 
A future experiment would be done on a cleaned track, 
done with more trials to reduce the effect of outliers, and done with an undamaged fan
that is secured to the cart and made sure not to fall. 

\end{document}