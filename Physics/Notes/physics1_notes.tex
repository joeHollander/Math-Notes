\documentclass{article}
\usepackage{amsfonts}
\usepackage{amsmath}
\usepackage{siunitx}

\title{AP Physics 1 Notes}
\author{Joe Hollander}
\date{2024-2025}

\begin{document}
\maketitle 

\section*{Forces and Newton's Laws}
\begin{itemize} 
    \item \textbf{Intertia} is a property of matter, 
and is the resistance to a change in motion (acceleration),
or a tendency to maintain a state of motion. 
Acceleration is any change is speed or direction.

    \item \textbf{Force} is any push or pull on an object (contact or non-contact).
    \item \textbf{Net Force} is the combination of, or the sum of, all simultaneous forces.
    \item \textbf{Dynamics} considers the forces that affect the motion of moving objects and systems. 
    \item \textbf{Magnetism} is the magnetic force between two wires carrying electric current. 
    \item \textbf{Field} is the ability to exert force over a distancev without contact.
    \item \textbf{Period} is the time for one revolution.  
\end{itemize}

Newton's first law, the law of inertia, states that an object at rest will remain at rest, 
and an object in motion will remain in motion at a constant velocity, unless acted upon by an unbalanced force.
The inertia of an object is measured by its mass. 

Newton's second law can first be considered as a proportion between acceleration and net force: 
\[
\mathbf{a \propto F_{net}}, 
\]
and the proportion between acceleration and mass:
\[
\mathbf{a} \propto \frac{1}{m}.
\]
Combining the proportionalities gives the equation form of Newton's second law:
\[
\mathbf{a} = \frac{\mathbf{F_{net}}}{m}, 
\]
or
\[
\mathbf{F_{net}} = m\mathbf{a}.
\]

Newton's third law, sometimes called the law of action/reaction, states that whenever one body exerts a force on a second body, 
the first body experiences a force that is equal in magnitude opposite in dirction to the force that it exerts.
Rephrased, all contact forces are interactions. 

Static equilibrium occurs when the net force on an object is zero.
To find the net force, consider Newton's second law in both the $x$ and $y$-direction. 
Additionally, force magnitudes are always positive. 

Gravity is a field. The force of gravity is inversely proportional to the square of the distance:
\[
F_g \propto m_1m_2 
\]
and directly proportional to the product of the masses:
\[
F_g \propto \frac{1}{r^2}
\]
Objects apply the same forces to each other, but the accelerations can be different.
The universal gravitation constant 
\[
G = 6.67 \times 10^{-11} \frac{\mathrm{Nm^2}}{\mathrm{kg^2}}.
\] 

The velocity of an object with circular motion is its cicrumfrence divided by the period.
In a circle, the centripetal force is the force towards the center of the circle,
and the centrifugal force is the force away from the center of the circle, caused by inertia.
Newton's second law applies to centripetal force too:
\[
\mathbf{F_c} = m\mathbf{a_c} = \frac{m\mathbf{v}^2}{r}.
\]
Counter-clockwise is the positive direction. 


\end{document}
