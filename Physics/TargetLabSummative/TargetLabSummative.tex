\documentclass{article}
\usepackage{amsfonts}
\usepackage{amsmath}
\usepackage{siunitx}

\title{Target Lab Summative}
\author{Joe Hollander}
\date{Oct 2024}

\begin{document}
\maketitle 


16 velocities were measured at the edge of the table. 
The mean of the measurements was approximately $1.071 \si{m\per\second}$ and the median was $1.073\si{m\per\second}$. 
Calculations were done using the average of the mean and median, which was $1.0716 \si{m\per\second}$, in an attempt to suppress the effects of outliers. 
First, finding how long the sphere will be in the air using the position equation :
\[
y_f = \frac{1}{2}at^2 + v_yt + y_0.
\]
Here, the $y$-component of velocity is zero,
the acceleration of gravity $a = -9.81 \si{m\per\second^2}$ (negative since positive $y$-direction is considered up), the starting position $y_0 = 0.915 \, \text{meters}$,
and the final $y$ position is $y_f = 0.145  \, \text{meters}$.
Therefore the equation is 
\[
0.145 = -4.905t^2 + 0.915, 
\]
and solving for t yields 
$t = \sqrt{0.77/4.905} \approx 0.40 \, \text{seconds}.$
This means that the marble will be in the air for approximately 0.40 seconds. 

Next, finding the distance the sphere will travel using the position equation again:
\[
x_f = \frac{1}{2}at^2 + v_xt + x_0.
\]
In this situation, the $x$-component of velocity is our initial velocity $v_x = 1.0716 \si{m\per\second}$,
the time in the air is the same as above, 
and the initial position $x_0 = 0 \, \text{meters}$ since we're measuring from the edge of the table.
The final position is therefore 
\[
x_f = 1.0716\sqrt{0.77/4.905} \approx 0.42 \, \text{meters}. 
\]

Placing the ring 42 cm away from the edge yielded the video attached. 

\medskip

Velocity Measurements (m/s):
1.051, 1.058, 1.063, 1.063, 1.065, 1.066, 1.069, 1.072, 1.073, 1.077, 1.077, 1.078, 1.078, 1.08, 1.081, 1.082


\end{document}