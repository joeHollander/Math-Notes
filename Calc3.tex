\documentclass{article}
\usepackage{graphicx}

\title{Calculus 3 notes}
\author{Joe Hollander}
\date{Summer 2024}

\begin{document}
\maketitle

\newtheorem{theorem}{Theorem}[section]

\section*{Vector Functions}
\noindent Dot product: 
\[a \cdot b = |a||b|cos(\theta).\]
Cross product:
To find the cross product of a and b find the
determinant of 3x3 matrix with unit vectors: $\mathbf{i, j, k}$
for the first row. Additionally: 
\[|a \times  b| = |a|b|sin(\theta).\]

\noindent The vector equation of a line starting at point $P$, $(a, b, c)$, 
with parralel vector $\vec{v}$, $\left\langle d, e, f\right\rangle$,
and paramater $t$:
\[
r(t) = P + t\vec{v} = 
\left\langle a + dt, b + et, c + ft\right\rangle.
\]
Given a constant point on the plane $P_0$, $(x_0,y_0,z_0)$
and a point on the plane $P$, $(x,y,z)$, 
the vector $\vec{P_0P}$ must be orthogonal to the normal vector of the plane $N$, 
$\left\langle a, b, c \right\rangle$.
Therefore, \[N \cdot \vec{P_0P} = 0, \]
and \[a(x-x_0) + b(y-y_0) + c(z-z_0) = 0.\]

\noindent Given vector function $r(t) = \left\langle f(t), g(t), h(t) \right\rangle$,
\[r'(t) = \left\langle f'(t), g'(t), h'(t) \right\rangle,\] and 
\[R(t) = \left\langle F(t), G(t), H(t) \right\rangle.\]

\noindent Length of curve from a to b is: \[\int_{a}^{b} |r'(t)| dt,\]
\[s(t) = \int_{0}^{t} |r'(x)| dx.\] An equation $y = f(x)$ can be paramaterized as
$r(t) = \left\langle t, f(t) \right\rangle$. The unit tangent vector: $T(t) = r'(t)/|r'(t)|$.
The principle unit normal vector: $N(t) = T'(t)/|T'(t)|$.
The binomial vector: $B = T \times N$. The osculating plane is
the one formed by $T$ and $N$. 
Curvature:
\[
\kappa(x) = |\frac{dT}{ds}| = \frac{|T'(t)|}{|r'(t)|}
= \frac{|r'(t) \times r''(t)|}{|r'(t)|^3}
\]
Torsion:
\[
\tau(x) = -N \cdot \frac{dB}{ds} 
= \frac{(r'(t) \times r''(t)) \cdot r'''(t)}{|r'(t) \cdot r''(t)|}.
\]

Working backwords from an acceleration vector, $\vec{a}$,
we can derive a position vector function of an object affected 
by gravity.
\[\vec{a} = \left\langle0, -9.8\right\rangle.\]
Given an initial velocity vector, $\vec{v_0}$ with speed, $s$,
and angle, $\alpha$: 
\[\vec{v_0} = \left\langle s\cos(\alpha), s\sin(\alpha)\right\rangle,\]
\[v(t) = \left\langle s\cos(\alpha), -9.8t + s\sin(\alpha)\right\rangle,\]
and
\[r(t) = \left\langle s\cos(\alpha)t, -4.9t^2 + s\sin(\alpha)t\right\rangle,\]
\[x = s\cos(\alpha)t, y = -4.9t^2 + s\sin(\alpha)t\].

\section*{Partial Derivatives}\
\
\noindent Domain for a function of two variables can be written as: 
\[D = \{(x,y)\:|\: x + y != 2, xy < -1\}.\]
Similar to contour drawings on maps, level curves of a function $f$ of two variables
are the curves with equations $f(x,y) = k$ where $k$ is a constant. 
\begin{center}
    \includegraphics[width=12cm,height=12cm, keepaspectratio]{LevelCurves.png}
\end{center}
Functions of three variables are the same as functions of two variables.
Functions of multiple variables can be packaged into a function of one vector
variable. For multi variable limits, the notation is:
\[\lim_{(x,y)\to (a,b)}f(x,y) = L.\]
Since the limit of a multi variable function can be approached from any way,
the definition of what makes a limit not exist has to be redefined. If 
$f(x,y) \to L_1$ as $(x,y) \to (a,b)$ along a path $C_1$ and $f(x,y) \to L_2$ as
$(x,y) \to (a,b)$ along a path $C_2$ and $L_1 \neq L_2$, then $\lim_{(x,y)\to (a,b)}f(x,y)$
does not exist. The easiest paths to evaluate are often the x and y-axis. 
If evaluating on the x-axis (or y-axis) substitute in $y=b$ (or $x=a$) and then solve
\[\lim_{(x,y) \to (a, b)} f(x,b),\] or \[\lim_{(x,y) \to (a, b)} f(a,y).\] 
Substituting in $y = mx$ is also an easy path to evaluate. A polynomial function of two variables
is in the form: \[f(x,y) = cx^n y^n.\] A rational polynomial is the ratio of two polynomials.
Since a polynomial is continous everywhere, a polynomial $p$ has the property:
\[\lim_{(x,y) \to (a,b)} p(x,y) = p(a,b).\]
Similarly a rational function $q$ has the same property provided $r(a,b) \neq 0$:
\[\lim_{(x,y) \to (a,b)} q(x,y) = \frac{p(a,b)}{r(a,b)} = q(a,b).\]
Squeeze theorem applies to multivariable functions also. Additionally,
the epsilon-delta definition of a limit is useful in proving that the limit is a value.
The definition states that for every $\epsilon > 0$ there exists a $\delta > 0$. 
$\delta$ bounds the domain such that:
\[0 < \sqrt{(x-a)^2 + (y-b)^2} < \delta,\]
and $\epsilon$ bounds the range such that: \[|f(x,y) - L| < \epsilon.\]
It's often easiest to start with the epsilon inequality first,
and finding a relation between delta and epsilon such as $\delta = \epsilon/3$
completes the proof. The epsilon-delta definition extends to vector functions:
\[0 < |\vec{x} - \vec{a}| < \delta, \] and
\[|f(\vec{x}) - L| < \epsilon.\]
A multivariable function is defined as continous if
\[\lim{(x,y) \to (a,b) f(x,y) = f(a,b)}.\] 
If two functions $f$ and $g$ are continuous then their composition function
$h = f \: \circ g = f(g(x,y))$ is also continous. Fix $y=b$ so that $g(x)=f(x,b)$.
The derivative of $f$ with respect to $x$ at $a$ is denoted by:
\[g'(a) = f_x(a,b) = \frac{\partial f}{\partial x}.\] Therefore, the limit definition becomes:
\[f_x(a,b) = \lim_{h \to 0}\frac{f(a+h,b) - f(a,b)}{h},\]
or \[f_y(a,b) = \lim_{h \to 0}\frac{f(a,b+h) - f(a,b)}{h}.\]

\noindent Visually: 
\begin{center}
    \includegraphics[width=12cm,height=12cm, keepaspectratio]{PartialDerivative.png}
\end{center}

When implicity differentiating equations with respect to $x$ such as:
\[x^3 + y^3 + z^3 + 6xyz = 0,\] where $z = f(x,y)$. Treat $y$ as a constant and $z$
as a function of $x$. Therefore, the derivative would be:
\[3x^2 + 3z^2\frac{\partial z}{\partial x} + 6yz + 6xy\frac{\partial z}{\partial x} = 0.\]
Partial derivatives of functions of three variables are the same. Second order
partial derivatives are denoted as:
\[(f_x)_x = \frac{\partial^2 f}{\partial x^2},\]
and partial derivatives of order $n$ are denoted as:
\[\frac{\partial^n f}{\partial x^n }.\]
Clairaut's theorem states that if $f$ is defined on a domain disk $D$ that contains
the point $(a,b)$ then
\[f_{xy}(a,b) = f_{yx}(a,b).\]




\end{document}